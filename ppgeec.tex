\documentclass[a4paper,12pt,openright,twoside]{book}
%\documentclass[a4paper,12pt,openany,oneside]{book}

% PACOTES OBRIGATÓRIOS

% Use estes pacotes para poder digitar diretamente as letras com acento
% e para que a hifenização funcione corretamente
\usepackage[utf8]{inputenc}
\usepackage{ae}
% Para usar fontes standard ao invés das do LaTeX (gera melhores PDFs)
\usepackage{pslatex}
% Para a hifenização em português
\usepackage[portuges, brazil]{babel}
% Para que os primeiros parágrafos das seções também sejam indentados
\usepackage{indentfirst}
% Para poder incluir gráficos (figuras)
\usepackage{graphicx}
% Para poder fazer glossário ou lista de símbolos
% Use a segunda opção se quiser incluir na definição do símbolo a
% página e/ou a equação onde ela foi definida
%ccze \usepackage[portuguese,noprefix]{nomencl}
%\usepackage[portuguese,noprefix,refeq,refpage]{nomencl}
% Para permitir espaçamento simples, 1 1/2 e duplo
%ccze \usepackage{setspace}
% Para usar alguns comandos matemáticos avançados muito úteis
\usepackage{amsmath}
% Para poder usar o ambiente "comment"
\usepackage{verbatim}
% Para poder ter tabelas com colunas de largura auto-ajustável
\usepackage{tabularx}
% Para executar um comando depois do fim da página corrente
\usepackage{afterpage}
% Para formatar URLs (endereços da Web)
\usepackage{url}
% Para reduzir os espaços entre os ítens (itemize, enumerate, etc.)
% Este pacote não faz parte da distribuição padrão do LaTeX.
\usepackage{lib/noitemsep}
% Para as citações bibliográficas
%\usepackage[alf]{abntex2cite}
\usepackage[abbr]{lib/harvard}	% As chamadas são sempre abreviadas
%\harvardparenthesis{square}	% Colchetes nas chamadas
%\harvardyearparenthesis{round}	% Parêntesis nos anos das referências
%\renewcommand{\harvardand}{e}	% Substituir "&" por "e" nas referências

% PACOTES OPCIONAIS

% Para poder incluir arquivos Postscript com cores (do Xfig, por exemplo)
\usepackage{color}
% Para ter células em tabelas que ocupam mais de uma linha
%ccze \usepackage{multirow}
% Para poder ter tabelas longas em mais de uma página
%\usepackage{longtable}
% Para poder escrever partes do texto em "n" colunas
%\usepackage{multicol}
% Se você quiser personalizar os cabeçalhos das páginas
%\usepackage{fancyheadings}
% Para incluir algoritmos e listagens de códigos
%\usepackage{listings}
% Para incluir pseudocódigos
%ccze \usepackage[portuguese, ruled, linesnumbered]{algorithm2e}
% Capítulos com títulos em um formato "decorado"
\usepackage{lib/capitulos}
% Hiperreferências dentro do texto e montagem dos links do índices dos
% para os leitores de pdf (deve ser o último pacote a ser inserido).
\usepackage[breaklinks]{hyperref}
% Referência correta de ambientes flutuantes (como figuras, tabelas e algoritmos).
\usepackage[all]{hypcap}



% NOVOS COMANDOS

% As definições dos novos comandos estão agrupadas no arquivo "comandos.tex"
% Esta separação é opcional: se você preferir, pode por as definições
% diretamente neste arquivo
\input{lib/comandos.tex}

%
% As margens
%

% Direção horizontal

% Margem interna
% Nas páginas ímpares
\setlength{\oddsidemargin}{3.5cm}         % Margem real desejada
% Nas páginas pares
\setlength{\evensidemargin}{2.5cm}        % Margem real desejada
% Largura do texto
\setlength{\textwidth}{15cm}
% As margens laterais no LaTeX são sempre 1 polegada maiores do que as
% fixadas. Se foi fixada \setlength{\oddsidemargin}{3.5cm}, a margem
% real seria de 3.5+2.54=6.04cm. Para permitir que você não tenha que
% fazer esta conta, pode usar o número desejado nas linhas anteriores
% e a gente subtrai 1in nas próximas linhas
\addtolength{\oddsidemargin}{-1in}
\addtolength{\evensidemargin}{-1in}
% Note que a margem direita não é fixada diretamente:
% ela é obtida subtraindo-se os outros valores da largura da página.
% 3.5+15+x=21cm (largura A4) -> x = margem externa = 2.5cm

% Direção vertical

% Margem superior (entre o topo da folha e o cabeçalho), altura do
% cabeçalho e distância entre o fim do cabeçalho e o início do texto
\setlength{\topmargin}{2.0cm}             % Margem real desejada
\setlength{\headheight}{1.0cm}
\setlength{\headsep}{1.0cm}
% Altura do texto (sem cabeçalho e rodapé)
\setlength{\textheight}{22.7cm}
% Distância do fim do texto ao rodapé
\setlength{\footskip}{1.0cm}
% A margem superior no LaTeX é sempre 1 polegada maior do que a
% fixada. Se foi fixada \setlength{\topmargin}{2.0cm}, a margem
%real seria de 2.0+2.54=4.54cm. Para permitir que você não tenha que
% fazer esta conta, pode usar o número desejado na linha anterior
% e a gente subtrai 1in na próxima linha
\addtolength{\topmargin}{-1in}
% Note que a margem inferior não é fixada diretamente:
% ela é obtida subtraindo-se os outros valores, sem incluir o
% "footskip", da altura da página.
% 2.0+1.0+1.0+22.7+x=29.7cm (altura A4) -> x = margem inferior = 3cm

%
% O estilo das referências bibliográficas
%

\bibliographystyle{bibliografia/ppgee}

%
% O espaçamento entre linhas
%

% As páginas iniciais são sempre em espaçamento simples
%ccze \singlespacing

% Para a criação do glossário (ou lista de símbolos)
%ccze \makenomenclature

% Lista de arquivos a serem processados. Estes comandos "includeonly" são
% dispensáveis e devem obrigatoriamente ser comentados na hora de gerar o
% documento definitivo. Eles servem para compilar apenas parte do documento.
% É útil, durante a redação, para que não se tenha de compilar todo o
% documento a cada vez que se faz uma alteração. Por exemplo, se eu estou
% fazendo alterações na dedicatória e as outras partes não têm interesse no
% momento, posso incluir (descomentar) a linha "\includeonly{preambulo}"
%\includeonly{rosto}
%\includeonly{catalograficos}
%\includeonly{preambulo}
%\includeonly{resumos}
%\includeonly{introducao/introducao}
%\includeonly{estilo/estilo}
%\includeonly{matematica/matematica}
%\includeonly{figuras/figuras}
%\includeonly{conclusao/conclusao}
%\includeonly{apendice/apendice}
%ccze 
%ccze % Inicia o texto
%ccze \begin{document}
%ccze 
%ccze % Paginas iniciais (sem numeração)
%ccze \pagestyle{empty}
%ccze 
%ccze % Página de rosto (capa interna)
%ccze %
% ********** Página de Rosto
%

% titlepage gera páginas sem numeração
\begin{titlepage}

\begin{center}

\small

% O comando @{} no ambiente tabular x é para criar um novo delimitador
% entre colunas que não a barra vertical | que é normalmente utilizada.
% O delimitador desejado vai entre as chaves. No exemplo, não há nada,
% de modo que o delimitador é vazio. Este recurso está sendo usado para
% eliminar o espaço que geralmente existe entre as colunas
\begin{tabularx}{@{}l@{}c@{}r@{}}
	\parbox[c]{3cm} &
	\begin{center}
		\textsf{Instituto do Cérebro}
	\end{center}
	\parbox[c]{2cm}
\end{tabularx}

%ccze \begin{tabularx}{\linewidth}{@{}l@{}C@{}r@{}}
%ccze % A figura foi colocada dentro de um parbox para que fique verticalmente
%ccze % centralizada em relação ao resto da linha
%ccze \parbox[c]{3cm}{\includegraphics[width=\linewidth]{pre-textuais/figuras/UFRN}} &
%ccze \begin{center}
%ccze \textsf{\textsc{Universidade Federal do Rio Grande do Norte\\
%ccze Centro de Tecnologia\\
%ccze Programa de Pós-Graduação em Engenharia Elétrica e de Computação}}
%ccze \end{center} &
%ccze \parbox[c]{2cm}{\includegraphics[width=\linewidth]{pre-textuais/figuras/PPgEE}}
%ccze \end{tabularx}

% O vfill é um espaço vertical que assume a máxima dimensão possível
% Os vfill's desta página foram utilizados para que o texto ocupe
% toda a folha
\vfill

\LARGE

\textbf{Células de velocidade no hipocampo de roedores são interneurônios}

\vfill

\Large

\textbf{José Henrique Targino Dias Góis}

\vfill

\normalsize

Orientador: Prof. Dr. Adriano Bretanha Lopes Tort
% Se não houver co-orientador, comente a próxima linha
% \\[2ex] Co-orientador: Prof. Dr. Beltrano Catandura do Amaral

\vfill

\hfill
\parbox{0.5\linewidth}{
	\textbf{Projeto de pesquisa}
	apresentado ao Programa de Pós-Graduação em Neurociências da UFRN
	(área de concentração: Neurocomputação, Neuroengenharia, Neuroterapia)
	como parte dos requisitos para obtenção do título de
	Doutor em Ciências.
}

\vfill

\large

% Este número de ordem deve ser obtido na coordenação do PPgEE
% Corresponde ao número seqüencial da sua tese ou dissertação:
% por exemplo, a 25ª tese de doutorado terá o número de ordem D25
% Evidentemente, este dado não existe para propostas de tema, caso
% em que a próxima linha deve ser comentada.
% Número de ordem PPgEEC: M000

Natal, RN, Março de 2018

\end{center}

\end{titlepage}

%ccze 
%ccze % Ficha catalográfica: os dados catalográficos devem ser fornecidos
%ccze % pela BCZM.
%ccze % Só são incluídos na versão final da tese ou dissertação. Não são
%ccze % incluídos nem na proposta de tema de qualificação nem na versão
%ccze % preliminar da tese ou dissertação: nestes casos, comente a próxima linha.
%ccze %
% ********** Ficha Catalográfica
%

\newpage

\begin{center}

% Aqui não se usou \vfill porque o \vfill é construído internamente com
% o comando \vspace. Espaços verticais no início da folha com \vspace
% são ignorados. Para que isto não ocorra deve-se usar o \vspace*
% \vspace*{\fill} é como se fosse um \vfill*
\vspace*{\fill}

Divisão de Serviços Técnicos\\[1ex]
Catalogação da publicação na fonte.
UFRN / Biblioteca Central Zila Mamede

\vspace{2ex}

\begin{tabular}{|p{0.9\linewidth}|} \hline
\\
Gois, José Henrique Targino Dias.\\
\hspace{1em} Código de velocidade no hipocampo dorsal/ Me. José Henrique Targino Dias Góis - Natal, RN, 2018 \\
\hspace{1em} xx p. \\
\\
\hspace{1em} Orientador: Prof. Adriano Bretanha Lopes Tort \\
%\hspace{1em} Co-orientador: Beltrano Catandura do Amaral \\
\\
\hspace{1em} Tese (doutorado) - Universidade Federal do Rio Grande do Norte.
Instituto do Cérebro. Programa de Pós-Graduação em Neurociências.\\
\\
\hspace{1em} 0. Tese. 1. Neurociências. 2. Comportamento animal. 3. Eletrofisiologia. 4. Estatística. 5. Estrutura de dados I. Título. \\
\\
RN/UF/BCZM \hfill CDU xxx.xxx(xxx.x) \\ \hline
\end{tabular} 

\end{center}

%ccze 
%ccze % Assinaturas da banca, dedicatória e agradecimentos
%ccze % Só são incluídos na versão final da tese ou dissertação. Não são
%ccze % incluídos nem na proposta de tema de qualificação nem na versão
%ccze % preliminar da tese ou dissertação: nestes casos, comente a próxima linha.
%ccze %
% ********** Página de assinaturas
%

\begin{titlepage}

\begin{center}

\LARGE

\textbf{Código de velocidade no hipocampo dorsal}

\vfill

\Large

\textbf{José Henrique Targino Dias Góis}

\end{center}

\vfill

% O \noindent é para eliminar a tabulação inicial que normalmente é
% colocada na primeira frase dos parágrafos
\noindent
% Descomente a opção que se aplica ao seu caso
% Note que propostas de tema de qualificação nunca têm preâmbulo.
Proposta de projeto de Doutorado
%Tese de Doutorado
aprovada em 01 de Abril de 2018 pela banca examinadora composta
pelos seguintes membros:

% Os nomes dos membros da banca examinadora devem ser listados
% na seguinte ordem: orientador, co-orientador (caso haja),
% examinadores externos, examinadores internos. Dentro de uma mesma
% categoria, por ordem alfabética

\begin{center}

\vspace{1.5cm}\rule{0.95\linewidth}{1pt}
\parbox{0.9\linewidth}{%
Prof. Dr. Adriano Bretanha Lopes Tort (orientador) \dotfill\ DCA/UFRN}

\vspace{1.5cm}\rule{0.95\linewidth}{1pt}
\parbox{0.9\linewidth}{%
Prof. Dr. Orientadora de Software(co-orientador) \dotfill\ DCA/UFRN}

\vspace{1.5cm}\rule{0.95\linewidth}{1pt}
\parbox{0.9\linewidth}{%
Prof. Dr. Diego Andreas LaPlagne \dotfill\ DCEP/UFFN}

\vspace{1.5cm}\rule{0.95\linewidth}{1pt}
\parbox{0.9\linewidth}{%
Profª Drª Orientador de Estatística \dotfill\ DCA/UFRN}

\end{center}

\end{titlepage}

%
% ********** Dedicatória
%

% A dedicatória não é obrigatória. Se você tem alguém ou algo que teve
% uma importância fundamental ao longo do seu curso, pode dedicar a ele(a)
% este trabalho. Geralmente não se faz dedicatória a várias pessoas: para
% isso existe a seção de agradecimentos.
% Se não quiser dedicatória, basta excluir o texto entre
% \begin{titlepage} e \end{titlepage}

\begin{titlepage}

\vspace*{\fill}

\hfill
\begin{minipage}{0.5\linewidth}
\begin{flushright}
\large\it
A Deus, por sermos um só corpo e um só espírito.
\end{flushright}
\end{minipage}

\vspace*{\fill}

\end{titlepage}

%
% ********** Agradecimentos
%

% Os agradecimentos não são obrigatórios. Se existem pessoas que lhe
% ajudaram ao longo do seu curso, pode incluir um agradecimento.
% Se não quiser agradecimentos, basta excluir o texto após \chapter*{...}

\chapter*{Agradecimentos}
\thispagestyle{empty}

\begin{trivlist}  \itemsep 2ex

\item A Deus, por ter me guiado na escuridão.

\item A minha avó, Maria de Lourdes Dias Góis, por ter sido instrumento de Deus.

\item Aos meus adversários que me fizeram forte.

\item A minha família por ter me dado motivos para seguir acreditando.

\item A República Federativa do Brasil e suas autarquias: UFRN e CAPES pelo apoio financeiro e estrutural.

\end{trivlist}

%ccze 
%ccze %
%ccze % O espaçamento entre linhas (ATENÇÃO A ESTA PARTE!)
%ccze %
%ccze %%%%%%%%%%%%%%%%%%%%%%%%%%%%%%%%%%%%%%%%%%%%%%%%%%%%%%%%%%%%%%%%%%%%%%%%%%%%
%ccze % PARA A VERSÃO FINAL:
%ccze % Deve ser usado espaçamento simples nas páginas de texto
%ccze \singlespacing
%ccze % PARA A QUALIFICAÇÃO E PARA A VERSÃO INICIAL:
%ccze % Deve ser usado espaçamento 1 1/2 nas páginas de texto
%ccze %\doublespacing
%ccze %%%%%%%%%%%%%%%%%%%%%%%%%%%%%%%%%%%%%%%%%%%%%%%%%%%%%%%%%%%%%%%%%%%%%%%%%%%%
%ccze 
%ccze % Resumo/Abstract
%ccze %
% ********** Resumo
%

% Usa-se \chapter*, e não \chapter, porque este "capítulo" não deve
% ser numerado.
% Na maioria das vezes, ao invés dos comandos LaTeX \chapter e \chapter*,
% deve-se usar as nossas versões definidas no arquivo comandos.tex,
% \mychapter e \mychapterast. Isto porque os comandos LaTeX têm um erro
% que faz com que eles sempre coloquem o número da página no rodapé na
% primeira página do capítulo, mesmo que o estilo que estejamos usando
% para numeração seja outro.
\mychapterast{Resumo}


Para explicar a capacidade de navegar no espaço, Edward Tollman postulou a existência de um mapa cognitivo no cérebro.
As neurociências, desde então, busca descrever os elementos cerebrais que participam na codificação dos elementos necessários para promover tal capacidade.
Neste projeto apresento os avanços realizados na caracterização de uma sub-população neural que participa na codificação da velocidade escalar do corpo do animal.
Analisando banco de dados abertos descobri a existência do código de velocidade presente na taxa de emissão de potenciais de ação nos neurônios do hipocampo dorsal.
Demonstrei que esse código independe da frequência de oscilação theta, que ele é estável ao longo do espaço e do tempo, e também é persistente a diferentes contextos.
A classsificação dos neurônios através de formato do potencial de ação, taxa de emissão de potenciais de ação, e a dependencia temporal dos potenciais de ação entre neurônios me proveu fortes indícios que esse código neural é presente apenas nos neurônios inibitórios.
Analisando os neurônios excitatórios na arena quadrada revelou que estes são modulados pela velocidade; o labirinto linear revelou a correlação de Pearson como um mal indíce de codificação de velocidade,os neurônios piramidais apresentaram uma acodificação espúria como subproduto de sua codificação do espaço.
Empreguei um modelo não linear para resolver a natureza etologia da interdependencia estatística de espaço e velocidade; esta análise me confirmou a prevalência do código de velocidade nos interneurônios e confirmou a hipótese da codificação espúria.
Os resultados preliminares desse projeto demonstraram que o código de velocidade está presente no hipocampo em uma subpopulação eletrofisiologicamente homogenea.
O presente projeto termina com uma proposta de uma ferramenta para armazenar e compartilhar informações e dados de experimentos eletrofisiológicos.

%Ainda, demonstro a estabilidade do código de velocidade ao longo do tempo e espaço e de contextos.
%Demonstro que essa subpopulação possui característica inibitória: eles são inibitórios, apresentam alta frequencia de potenciais de ação e possuem formato de onda curto.
%Demonstro que os neurônios pijj 
%A doutrina neural põe o neurônio como a unidade básica do sistema nervoso central, que por sua vez é o sistema responsável pela coordenação do comportamento animal. Os neurônios possuem em sua membrana um complexo proteico que da a capacidade de excitação. Sabe-se que somos capazes de inferir o comportamento animal através de observer a taxa em que o neurônio alterna entre excitação e repouso. Nesse trabalho neurônios temos interesse analisar a entra
%Navegação espacial é uma capacidade presente desde os artropodes até os humanos.
%Teóricamente essa capacidade depende do cérebro ser capaz de construir uma representação de si e do espaço afim de construir uma memória de sequência.
%O sistema mesolimbico foi demonstrado ser necessário para a expressão dessa capacidade.



\vspace{1.5ex}

{\bf Palavras-chave}: Hipocampo, eletrofisiolgia, interneurônios, velocidade, comportamento.

\mychapterast{Abstract}
On the journey to explain the spatial navigation capability of animals, Edward Tollman postulated the existence of a cognitive brain map.
The neurosciences thenceforth describe cerebral elements providers of the cognitive functions that promote this capability.
The present project presents advancements made on the characterization of a neuronal population that represent the animal body linear velocity.
Analyzing open database I found the existence of the speed code on the action potential emition rate of the dorsal hippocampal neurons.
Furthermore, I demonstrated that this rate code is independent of local field theta oscillation; and the code is stable over space and time and persistent over contexts.
The analyses of the waveform shape, action potential rate, and the temporal relatitonship of action potentials revealed the prevalence of the speed code on the rate of action potentials emited by inhibitory neurons.
The latter finding contradicts the literature, deeper analysis revealed the interpenendence of the speed coding and spatial coding in principal neurons. Therefore, I hypothesized that speed coding in excitatory neurons is a by-product of the spatial coding.
Simulated data of speed, conjuctive and spatial code proved me right, The space vs speed covariance is higher in linear rather than in square arena, this rose higher speed scores in linear arena. 
To solve this ethologycal bias I utilized multivariate exponential mixture model that model simulated firing rate as a mixture of spatial and speed. The speed over space log-likellihood ratio is the metric I propose as a substitute of the current speed score.
Applying this ratio metric in the real data I reinforce the inhibitory nature of speed coding in the dorsal hippocampus and I deny the excitatory one. However strong, I still need to investigate the conjuctive code of space and speed. Which shall be matter of investigation of the following days.

\vspace{1.5ex}

{\bf Keywords}: Hippocampus, Interneurons, Speed coding, Spatial navigation, Speed cells.

%ccze 
%ccze % Paginas introdutórias (com numeração romana)
%ccze \frontmatter
%ccze 
%ccze % Lista de conteúdo (sumário, gerado automaticamente)
%ccze % Aqui, e em todos os itens antes da introdução, o comando \phantomsection é utilizado.
%ccze % O seu uso é neecssário para auxiliar o pacote "hyperref" a fazer a referência correta
%ccze % dos links do sumário, das listas (de tabelas, figuras, algoritmos) com as páginas 
%ccze % respectivas.
%ccze % Caso seja tirado, o "hyperref" irá apontar o link do sumário para o abstract, o link
%ccze % do sumário para a lista de figuras, o link das listas de figuras para a lista de tabelas,
%ccze % e assim sucessivamente.
%ccze \phantomsection
%ccze \addcontentsline{toc}{chapter}{Sumário}
%ccze \tableofcontents
%ccze 
%ccze % Lista de figuras (gerada automaticamente)
%ccze \cleardoublepage
%ccze \phantomsection
%ccze \addcontentsline{toc}{chapter}{Lista de Figuras}
%ccze \listoffigures
%ccze 
%ccze % Lista de tabelas (gerada automaticamente)
%ccze \cleardoublepage
%ccze \phantomsection
%ccze \addcontentsline{toc}{chapter}{Lista de Tabelas}
%ccze \listoftables
%ccze 
%ccze % Glossário (gerado automaticamente - veja entradas em
%ccze % tex/00-introducao/introducao.tex e em tex/02-estilo/estilo.tex)
%ccze \cleardoublepage
%ccze \phantomsection
%ccze \renewcommand{\nomname}{Lista de Símbolos e Abreviaturas}
%ccze \markboth{\MakeUppercase{\nomname}}{\MakeUppercase{\nomname}}
%ccze \addcontentsline{toc}{chapter}{\nomname}
%ccze % O argumento opcional do comando \printnomenclature é a largura deixada
%ccze % para os símbolos no glossário. Se seus símbolos são "largos", como
%ccze % neste exemplo,  é melhor por mais espaço do que o 1cm que é reservado
%ccze % por default
%ccze \printnomenclature[1.5cm]
%ccze 
%ccze % Páginas do texto principal (com cabeçalho)
%ccze \mainmatter
%ccze \pagestyle{headings}
%ccze 
%ccze % Para facilitar a organização, foi criado um diretório para cada
%ccze % capítulo do documento, pois assim os arquivos das figuras ficam
%ccze % classificados por capítulos
%ccze 
%ccze % Cap. 1 - Introdução
%ccze %%
%% Capítulo 1: Modelo de Capítulo
%%

% Está sendo usando o comando \mychapter, que foi definido no arquivo
% comandos.tex. Este comando \mychapter é essencialmente o mesmo que o
% comando \chapter, com a diferença que acrescenta um \thispagestyle{empty}
% após o \chapter. Isto é necessário para corrigir um erro de LaTeX, que
% coloca um número de página no rodapé de todas as páginas iniciais dos
% capítulos, mesmo quando o estilo de numeração escolhido é outro.
\mychapter{Introdução}
\label{Cap:introducao}

Entender as variáveis que influenciam o comportamento animal é umas das fronteiras epistemológicas dos humanos.
Há inúmeras hipóteses de quais seriam os motores que levam os animais se comportarem como se comportam.
Assumindo a perspectiva de neurocientista eu presumo que o comportamento é gerado a partir de nosso corpo e é coordenado pelo nosso sistema nervoso central.
%Assumimos que o leitor possui conhecimento básico de eletrofisiologia celular.

Neste capítulo introdutório apresentarei uma breve revisão bibliográfica em neurociências sobre o comportamento de navegação espacial, afim de preparar o leitor aos objetivos específicos desse projeto de doutorado.

\section{Os animais}

O planeta Terra é palco de diversas formas de vida, dentre elas os animais.
%Os animais são dotados de corpos, estes são ao mesmo tempo o maior mistério do planeta como as estruturas mais admiraveis nele contido.
Os animais são dotados de corpos, estes são estruturas admiraveis que inspirou os humanos a construírem diversas ferramentas.
Os corpos são formados por células, que por sua vez são formado por moléculas, que são formadas por átomos, que são formados por partículas que são também fronteira epistemologica humana.
As células se especializam em diferentes tipos celulares, e se organizam formando orgãos.
Os orgãos são estruturas especializadas em capacitar os animais a desempenhar uma determinada função.
Por sua vez, os orgãos se conectam e unem-se para formar sistemas.

Embora os animais possuam corpos diferentes uns dos outros eles comungam de sistemas básicos.
Os sistemas presentes em todas as espécies de animais são chamados de sistema: respiratório, digestivo, excretor, circulatório, imune e nervoso.
Este ultimo pode se divide-se em duas partes, o sistema motor que os permitem locomover pelo movimento coordenado de seus músculos; e o sistema sensorial que os permite sentir o ambiente e seus corpos.
Os animais nascem dotados de instintos, padrões de movimento que são herdados pelos genitores.
Eles também são dotados de flexibilidade de comportamento, novos padrões de movimentos são gerados a medida que o animal experencia a vida.
%https://allpsych.com/psychology101/sensation_perception/

O sistema nervoso promove a integração entre capacidade de mover e de sentir, sendo responsável pela capacidade de perceber.
Percepção é a capacidade de associar qualidade a padrões de sentidos.
Não é mais uma simples transdução fisioquímica para transmissão absorção do sinal do mundo, mas pela significação do padrão.
Às retenções de padrões comportamentais e perceptuais damos os nomes de memória.

\section{Navegação espacial}

Há inúmeros contextos interessantes para estudar memória, neste projeto nós consideraremos a capacidade de navegação espacial.
%Dentre inúmeros contextos em que é interessante estudar a retenção dessas memórias mediante a experiência de vida, a navegação espacial é a que será estudada nesse projeto.
A navegação espacial é a capacidade que os animais possuem de se locomover no espaço com um propósito.
Para que isso seja possível o supõe-se que o animal seja capaz de se locomover no planeta e que seja capaz de atribuir qualidade às sensações.
Qualidades como um padrão sensorial é uma localidade A onde possui a qualidade X e outro é uma localidade B onde encontro qualidade Y.

\subsection{Migração}
Essa capacidade é notória em diversas espécies de animais: pássaros, peixes, insetos, mamíferos.
Estas espécies de animais possuem inúmeras raças as quais seus membros se deslocam sasonalmente por longas distâncias para locais já conhecidos, este coportamento chama-se migração.
A migração é sempre realizada com um propósito, seja alimentação, abrigo de adversidades climáticas ou ambiente propício ao acasalmento.

\subsection{Estratégias migratória}

%Os seus defensores mais conhecidos foram Dr. Burrhus Skinner e o Dr. Edward Tolman, respectivamente.
As neurociências foi palco de um grande embate ente as teorias do estímulo-resposta e a cognitiva; especialmente no que se refere à navegação espacial.
%A libertação de uma teoria antiga foi um alívio para uma sociedade mal instruida, que adotou métodos punitivos - como o palmatório - numa tentativa infeliz de aumentar o aprendizado.
Voltando no tempo, na década de 1940 duas teorias competiram pelo modelo do comportamento animal, a teoria do estímulo-resposta e a cognitiva.
Enquanto uma afirmava que o comportamento seria controlada pelo cérebro como um reflexo direto da memória de estímulos, a outra defendia que formaria-se a experiência formaria uma mudança que representaria a valoração dos estímulos.

\subsection{Mapas cognitivos}
Felizmente Edward Tolman, enxergou o experimento que poderia refutar uma das hipóteses.\

houve Durante uma época conturbada na neurociências, em uma das respostas aos behavioristas Edward Tollman publicou um artigo  fortes embates Edward Tollman 

Algumas hipóteses foram levantadas sobre os mecanismos capazes de prover essa habilidade nos animais, citarei algumas delas abaixo.


%The distance which we traveled day by day was at first determined by dead reckoning, to be verified later by observations for latitude. The position of the ship was found, and marked on the chart, and the "dead reckoning" compared with the result obtained by calculation. The sketch was made solely by dead reckoning.

Formigas
migra
capacidade de 
experiências vividas nesse 
Opção dos padrões do mundo são chamadas
animal assim como a capacidade sensorial do animal é o listing 
padrões de movimento herdado passa a gerar novos padrões.
Os instintos básicos são dor
são dotados de instintos básicos que são os primeiros motores de seus comportamentos.
A capacidade de sentir o ambiente chamamos de tem eles encontram ambientes distintos, pela própria natureza do nosso planeta.
Em cada ambiente diferente o planeta expõe os animais a
Os animais tem a propriedade de modificar o seu comportamento pelas experiências as quais foram submetidas, a essa propriedade chamaremos de memória.
Lugares onde eles depositam confiança e criam lares, em outros desafios e depositam medo.


\section{Hipocampo e cognição}

\subsection{Memória declarativa}

\subsection{Estudos de lesão}

\subsection{Labirinto de Morris}


\section{Eletrofisiologia}
Onde reviso superficialmente aspéctos relevantes as hipóteses de trabalho deste projeto: a bioquímica da retenção e transmissão de informação no sistema nervoso. 

\subsection{Membranas celulares}
Nota do autor: Isso tudo tem Berg Tymoczko Stryker

O que são

De que são formadas

Propriedades das membranas

As células, as quais são as bases estruturais dos nossos organismos, são delimitadas e talvez até mesma definidas pela membrana celular.
Ela é a fronteira que define dois conjuntos aquosos, os que compõe o interior da célula e o que compõe o exterior da célula.
Embora fuja do escopo desse projeto, vale lembrar que essa mesma configuração molecular é também responsável por definir os limites das estruturas sub-celulares, motivo do plural da sub-sessão.
Embora fuja do escopo desse projeto, vale lembrar que essa mesma configuração molecular é também responsável por definir os limites das estruturas sub-celulares.

A natureza química dela é orgânica, uma definição circular, afinal é dita aos elementos que compõe os organismos.
São formadas por apenas duas moléculas de espessura, medindo entre 6 e 10 nanometros. 
Sua porção hidrofóbica é fica na porção central e 
Mais precisamente ela é formada por uma camada fosfo-lipídica, comportando-se como um colóide, assemelhando-se a uma bolha de sabão.
A sua montagem é realizada através de ligações iônicas, o que promove a característica flúidica.

Estrutura Pelo fato de que os ácidos graxos se ionizarem em $pH^{+}$ fisiológico A parte 

Em condições normais, há três tipos de fosfolipídios na membranas.

\subsection{DNA e proteínas}
Núcleo e o DNA

Ribossomos

Transcrição

Tradução

Transporte

Proteínas responsáveis

Canais iônicos

\subsection{Potencial de reversão}
Difusão e Osmose Lei de Fick

Nernst

\subsection{Potencial de ação}
Alan Hodgkin and Huxley

Axônio gigante da lula

Vesículas sinápticas

Neurotransmissores e seus receptores

Resposta pós-sinaptica

\subsection{Potencial pós-sinaptico}



\subsection{Codificação neural}
Juntamente ao 

\subsection{Arquitetura hipocampal}
Hipocampo e o sistema limbico

Giro denteado

Conexões subiculum

Pós subiculum

Córtex entorhinal

\section{Células de navegação}
Onde eu trago uma revisão do estado da arte na codificação da navegação espacial.

\subsection{Células de lugar}
O primeiro indício de um mapa cognitivo capaz de ser decodificado pelos humanos foi encontrado no hipocampo dorsal de roedores.
Em XXXX John O'Keefe 


\subsection{Células de direção}
Quem quando como e onde

Propriedade de disparo das células

Contexto onde elas foram experimentadas 

Conclusões sobre essas células

\subsection{Células de fronteira}
Quem quando como e onde

Propriedade de disparo das células

Contexto onde elas foram experimentadas 

Conclusões sobre essas células


\subsection{Células de grade}
Quem quando como e onde

Propriedade de disparo das células

Contexto onde elas foram experimentadas 

Conclusões sobre essas células


\subsection{Células de velocidade}
Quem quando como e onde

Propriedade de disparo das células

Contexto onde elas foram experimentadas 

Conclusões sobre essas células

%ccze 
%ccze % Cap. 2 - Teoria (Referencial Teórico)
%ccze \include{textuais/02-teoria/teoria}
%ccze 
%ccze % Cap. 3 - Trabalhos Relacionados
%ccze \include{textuais/03-trabalhos_relacionados/trabalhos_relacionados}
%ccze 
%ccze % Cap. 4 - Problema
%ccze \include{textuais/04-problema/problema}
%ccze 
%ccze % Cap. 5 - Implementação
%ccze \include{textuais/05-implementacao/implementacao}
%ccze 
%ccze % Cap. 6 - Experimentos e Resultados
%ccze \include{textuais/06-experimentos_e_resultados/experimentos_e_resultados}
%ccze 
%ccze % Cap. 7 - Conclusão
%ccze %%
%% Capítulo 5: Conclusões
%%

\mychapter{Conclusões}
\label{Cap:conclusao}

O capítulo final depende do tipo de documento. Nas propostas de tema
deve ser apresentado de forma clara e sucinta o assunto a ser
desenvolvido e o cronograma de execução do trabalho. Nas teses e
dissertações devem ser ressaltadas as principais contribuições do
trabalho e as suas limitações.

As contribuições devem evitar as adjetivações e julgamentos de valor.
Quanto às limitações, não tenha medo de as apresentar: é muito mais
reconhecido um autor que apresenta os casos em que sua proposta não se
aplica do que outro que parece não ter consciência deles.

\section{Células de velocidade são interneuronios}

\section{Correlação de Pearson é uma métrica inadequada}

\section{Razão entre os códigos de espaço e velocidade é adequado}


%ccze 
%ccze % Referências bibliogáficas (geradas automaticamente)
%ccze % Aqui, o comando \phantomsection é utilizado para auxiliar o pacote "hyperref" a fazer a
%ccze % referência correta dos links das referências bibliográficas com a página respectiva.
%ccze % Caso seja tirado, o "hyperref" irá apontar o link das referências bibliográficas para a
%ccze % última subseção da conclusão.
%ccze \phantomsection
%ccze \addcontentsline{toc}{chapter}{Referências bibliográficas}
%ccze \bibliography{bibliografia/bibliografia}
%ccze 
%ccze \appendix

%ccze %Apêndice A
%ccze \include{textuais/apendice/apendice}

\end{document}
