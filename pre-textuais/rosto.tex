%
% ********** Página de Rosto
%

% titlepage gera páginas sem numeração
\begin{titlepage}

\begin{center}

\small

% O comando @{} no ambiente tabular x é para criar um novo delimitador
% entre colunas que não a barra vertical | que é normalmente utilizada.
% O delimitador desejado vai entre as chaves. No exemplo, não há nada,
% de modo que o delimitador é vazio. Este recurso está sendo usado para
% eliminar o espaço que geralmente existe entre as colunas
\begin{tabularx}{@{}l@{}c@{}r@{}}
	\parbox[c]{3cm} &
	\begin{center}
		\textsf{Instituto do Cérebro}
	\end{center}
	\parbox[c]{2cm}
\end{tabularx}

%ccze \begin{tabularx}{\linewidth}{@{}l@{}C@{}r@{}}
%ccze % A figura foi colocada dentro de um parbox para que fique verticalmente
%ccze % centralizada em relação ao resto da linha
%ccze \parbox[c]{3cm}{\includegraphics[width=\linewidth]{pre-textuais/figuras/UFRN}} &
%ccze \begin{center}
%ccze \textsf{\textsc{Universidade Federal do Rio Grande do Norte\\
%ccze Centro de Tecnologia\\
%ccze Programa de Pós-Graduação em Engenharia Elétrica e de Computação}}
%ccze \end{center} &
%ccze \parbox[c]{2cm}{\includegraphics[width=\linewidth]{pre-textuais/figuras/PPgEE}}
%ccze \end{tabularx}

% O vfill é um espaço vertical que assume a máxima dimensão possível
% Os vfill's desta página foram utilizados para que o texto ocupe
% toda a folha
\vfill

\LARGE

\textbf{Células de velocidade no hipocampo de roedores são interneurônios}

\vfill

\Large

\textbf{José Henrique Targino Dias Góis}

\vfill

\normalsize

Orientador: Prof. Dr. Adriano Bretanha Lopes Tort
% Se não houver co-orientador, comente a próxima linha
% \\[2ex] Co-orientador: Prof. Dr. Beltrano Catandura do Amaral

\vfill

\hfill
\parbox{0.5\linewidth}{
	\textbf{Projeto de pesquisa}
	apresentado ao Programa de Pós-Graduação em Neurociências da UFRN
	(área de concentração: Neurocomputação, Neuroengenharia, Neuroterapia)
	como parte dos requisitos para obtenção do título de
	Doutor em Ciências.
}

\vfill

\large

% Este número de ordem deve ser obtido na coordenação do PPgEE
% Corresponde ao número seqüencial da sua tese ou dissertação:
% por exemplo, a 25ª tese de doutorado terá o número de ordem D25
% Evidentemente, este dado não existe para propostas de tema, caso
% em que a próxima linha deve ser comentada.
% Número de ordem PPgEEC: M000

Natal, RN, Março de 2018

\end{center}

\end{titlepage}
