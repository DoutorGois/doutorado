%%
%% Capítulo 1: Modelo de Capítulo
%%

% Está sendo usando o comando \mychapter, que foi definido no arquivo
% comandos.tex. Este comando \mychapter é essencialmente o mesmo que o
% comando \chapter, com a diferença que acrescenta um \thispagestyle{empty}
% após o \chapter. Isto é necessário para corrigir um erro de LaTeX, que
% coloca um número de página no rodapé de todas as páginas iniciais dos
% capítulos, mesmo quando o estilo de numeração escolhido é outro.
\mychapter{Introdução}
\label{Cap:introducao}

Entender as variáveis que influenciam o comportamento animal é umas das fronteiras epistemológicas dos humanos.
Há inúmeras hipóteses de quais seriam os motores que levam os animais se comportarem como se comportam.
Assumindo a perspectiva de neurocientista eu presumo que o comportamento é gerado a partir de nosso corpo e é coordenado pelo nosso sistema nervoso central.
%Assumimos que o leitor possui conhecimento básico de eletrofisiologia celular.

Neste capítulo introdutório apresentarei uma breve revisão bibliográfica em neurociências sobre o comportamento de navegação espacial, afim de preparar o leitor aos objetivos específicos desse projeto de doutorado.

\section{Os animais}

O planeta Terra é palco de diversas formas de vida, dentre elas os animais.
%Os animais são dotados de corpos, estes são ao mesmo tempo o maior mistério do planeta como as estruturas mais admiraveis nele contido.
Os animais são dotados de corpos, estes são estruturas admiraveis que inspirou os humanos a construírem diversas ferramentas.
Os corpos são formados por células, que por sua vez são formado por moléculas, que são formadas por átomos, que são formados por partículas que são também fronteira epistemologica humana.
As células se especializam em diferentes tipos celulares, e se organizam formando orgãos.
Os orgãos são estruturas especializadas em capacitar os animais a desempenhar uma determinada função.
Por sua vez, os orgãos se conectam e unem-se para formar sistemas.

Embora os animais possuam corpos diferentes uns dos outros eles comungam de sistemas básicos.
Os sistemas presentes em todas as espécies de animais são chamados de sistema: respiratório, digestivo, excretor, circulatório, imune e nervoso.
Este ultimo pode se divide-se em duas partes, o sistema motor que os permitem locomover pelo movimento coordenado de seus músculos; e o sistema sensorial que os permite sentir o ambiente e seus corpos.
Os animais nascem dotados de instintos, padrões de movimento que são herdados pelos genitores.
Eles também são dotados de flexibilidade de comportamento, novos padrões de movimentos são gerados a medida que o animal experencia a vida.
%https://allpsych.com/psychology101/sensation_perception/

O sistema nervoso promove a integração entre capacidade de mover e de sentir, sendo responsável pela capacidade de perceber.
Percepção é a capacidade de associar qualidade a padrões de sentidos.
Não é mais uma simples transdução fisioquímica para transmissão absorção do sinal do mundo, mas pela significação do padrão.
Às retenções de padrões comportamentais e perceptuais damos os nomes de memória.

\section{Navegação espacial}

Há inúmeros contextos interessantes para estudar memória, neste projeto nós consideraremos a capacidade de navegação espacial.
%Dentre inúmeros contextos em que é interessante estudar a retenção dessas memórias mediante a experiência de vida, a navegação espacial é a que será estudada nesse projeto.
A navegação espacial é a capacidade que os animais possuem de se locomover no espaço com um propósito.
Para que isso seja possível o supõe-se que o animal seja capaz de se locomover no planeta e que seja capaz de atribuir qualidade às sensações.
Qualidades como um padrão sensorial é uma localidade A onde possui a qualidade X e outro é uma localidade B onde encontro qualidade Y.

\subsection{Migração}
Essa capacidade é notória em diversas espécies de animais: pássaros, peixes, insetos, mamíferos.
Estas espécies de animais possuem inúmeras raças as quais seus membros se deslocam sasonalmente por longas distâncias para locais já conhecidos, este coportamento chama-se migração.
A migração é sempre realizada com um propósito, seja alimentação, abrigo de adversidades climáticas ou ambiente propício ao acasalmento.

\subsection{Estratégias migratória}

%Os seus defensores mais conhecidos foram Dr. Burrhus Skinner e o Dr. Edward Tolman, respectivamente.
As neurociências foi palco de um grande embate ente as teorias do estímulo-resposta e a cognitiva; especialmente no que se refere à navegação espacial.
%A libertação de uma teoria antiga foi um alívio para uma sociedade mal instruida, que adotou métodos punitivos - como o palmatório - numa tentativa infeliz de aumentar o aprendizado.
Voltando no tempo, na década de 1940 duas teorias competiram pelo modelo do comportamento animal, a teoria do estímulo-resposta e a cognitiva.
Enquanto uma afirmava que o comportamento seria controlada pelo cérebro como um reflexo direto da memória de estímulos, a outra defendia que formaria-se a experiência formaria uma mudança que representaria a valoração dos estímulos.

\subsection{Mapas cognitivos}
Felizmente Edward Tolman, enxergou o experimento que poderia refutar uma das hipóteses.\

houve Durante uma época conturbada na neurociências, em uma das respostas aos behavioristas Edward Tollman publicou um artigo  fortes embates Edward Tollman 

Algumas hipóteses foram levantadas sobre os mecanismos capazes de prover essa habilidade nos animais, citarei algumas delas abaixo.


%The distance which we traveled day by day was at first determined by dead reckoning, to be verified later by observations for latitude. The position of the ship was found, and marked on the chart, and the "dead reckoning" compared with the result obtained by calculation. The sketch was made solely by dead reckoning.

Formigas
migra
capacidade de 
experiências vividas nesse 
Opção dos padrões do mundo são chamadas
animal assim como a capacidade sensorial do animal é o listing 
padrões de movimento herdado passa a gerar novos padrões.
Os instintos básicos são dor
são dotados de instintos básicos que são os primeiros motores de seus comportamentos.
A capacidade de sentir o ambiente chamamos de tem eles encontram ambientes distintos, pela própria natureza do nosso planeta.
Em cada ambiente diferente o planeta expõe os animais a
Os animais tem a propriedade de modificar o seu comportamento pelas experiências as quais foram submetidas, a essa propriedade chamaremos de memória.
Lugares onde eles depositam confiança e criam lares, em outros desafios e depositam medo.


\section{Hipocampo e cognição}

\subsection{Memória declarativa}

\subsection{Estudos de lesão}

\subsection{Labirinto de Morris}


\section{Eletrofisiologia}
Onde reviso superficialmente aspéctos relevantes as hipóteses de trabalho deste projeto: a bioquímica da retenção e transmissão de informação no sistema nervoso. 

\subsection{Membranas celulares}
Nota do autor: Isso tudo tem Berg Tymoczko Stryker

O que são

De que são formadas

Propriedades das membranas

As células, as quais são as bases estruturais dos nossos organismos, são delimitadas e talvez até mesma definidas pela membrana celular.
Ela é a fronteira que define dois conjuntos aquosos, os que compõe o interior da célula e o que compõe o exterior da célula.
Embora fuja do escopo desse projeto, vale lembrar que essa mesma configuração molecular é também responsável por definir os limites das estruturas sub-celulares, motivo do plural da sub-sessão.
Embora fuja do escopo desse projeto, vale lembrar que essa mesma configuração molecular é também responsável por definir os limites das estruturas sub-celulares.

A natureza química dela é orgânica, uma definição circular, afinal é dita aos elementos que compõe os organismos.
São formadas por apenas duas moléculas de espessura, medindo entre 6 e 10 nanometros. 
Sua porção hidrofóbica é fica na porção central e 
Mais precisamente ela é formada por uma camada fosfo-lipídica, comportando-se como um colóide, assemelhando-se a uma bolha de sabão.
A sua montagem é realizada através de ligações iônicas, o que promove a característica flúidica.

Estrutura Pelo fato de que os ácidos graxos se ionizarem em $pH^{+}$ fisiológico A parte 

Em condições normais, há três tipos de fosfolipídios na membranas.

\subsection{DNA e proteínas}
Núcleo e o DNA

Ribossomos

Transcrição

Tradução

Transporte

Proteínas responsáveis

Canais iônicos

\subsection{Potencial de reversão}
Difusão e Osmose Lei de Fick

Nernst

\subsection{Potencial de ação}
Alan Hodgkin and Huxley

Axônio gigante da lula

Vesículas sinápticas

Neurotransmissores e seus receptores

Resposta pós-sinaptica

\subsection{Potencial pós-sinaptico}



\subsection{Codificação neural}
Juntamente ao 

\subsection{Arquitetura hipocampal}
Hipocampo e o sistema limbico

Giro denteado

Conexões subiculum

Pós subiculum

Córtex entorhinal

\section{Células de navegação}
Onde eu trago uma revisão do estado da arte na codificação da navegação espacial.

\subsection{Células de lugar}
O primeiro indício de um mapa cognitivo capaz de ser decodificado pelos humanos foi encontrado no hipocampo dorsal de roedores.
Em XXXX John O'Keefe 


\subsection{Células de direção}
Quem quando como e onde

Propriedade de disparo das células

Contexto onde elas foram experimentadas 

Conclusões sobre essas células

\subsection{Células de fronteira}
Quem quando como e onde

Propriedade de disparo das células

Contexto onde elas foram experimentadas 

Conclusões sobre essas células


\subsection{Células de grade}
Quem quando como e onde

Propriedade de disparo das células

Contexto onde elas foram experimentadas 

Conclusões sobre essas células


\subsection{Células de velocidade}
Quem quando como e onde

Propriedade de disparo das células

Contexto onde elas foram experimentadas 

Conclusões sobre essas células
