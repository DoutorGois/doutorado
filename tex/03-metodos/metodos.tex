%%
%% Capítulo 2: Regras gerais de estilo
%%

\mychapter{Métodos}
\label{Cap:methods}

Este capítulo apresenta considerações de ordem geral sobre a
organização que deve ser adotada no seu documento, tais como número de
páginas, margens e subdivisões.

\section{Dados abertos}
\label{Sec:opendata}

Eletrofisiologia Para verificar a existência do código de velocidade no hipocampo dors

\section{CRCNS/hc-3}
\label{Sec:crcns_hc3}

\section{Sumário dos dados}
\label{Sec:datasummary}

\section{Pre-processamento}
\label{Sec:preprocess}


\begin{lstlisting}
#!/bin/python
import random
import string
import time

def mkpass(size=16):
	chars = []
	chars.extend([i for i in string.ascii_letters])
	chars.extend([i for i in string.digits])
	chars.extend([i for i in '\'"!@#\$%*()-_=+[{}]~^,<.>;:/?'])

	passwd = ''

	for i in range(size):
		passwd += chars[random.randint(0, len(chars) - 1)]
		random.seed = int(time.time())
		random.shuffle(chars)
	return pass
\end{lstlisting}



\section{Código de velocidade}
\label{Sec:speedscore}
\begin{minted}[bgcolor=cyan!10]{python}
/**
* comentario
*/
public class HelloWorldApp {
public static void main (String argv[]){
  // Comentario
    System.out.println("Hello World!");
	}
}
\end{minted}

\section{Modelos não-lineares}
\label{Sec:speedscore}

