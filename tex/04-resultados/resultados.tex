%%

\mychapter{Resultados}
\label{Cap:results}

\section{Código de velocidade nos potenciais de ação do hipocampo}

\section{Código de velocidade independe da oscilação Theta}

\section{Código de velocidade é estável no tempo e no espaço}

\section{Código de velocidade é persistente nas tarefas}

\section{Decodificador linear da velocidade pelos potenciais de ação}

\section{Código de velocidade não é homogeneo nos neurônios}

\section{Código de velocidade é presente somente nos neurônios inibitórios}

\section{Neurônios excitatórios são facilitados pela velocidade linear}

\section{Neurônios excitatórios codificam velocidade na arena linear}

\section{O campo receptivo de lugar se confunde com a sintona de velocidade}

\section{Simulação mostra que a confusão entre espaço e velocidade é empírica}

\section{Mistura de modelos não lineares de múltiplas variáveis resolve a confusão}

\section{O coeficiente de velocidade não é adequado para código de velocidade}




