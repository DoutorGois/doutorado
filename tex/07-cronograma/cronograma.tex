%%
%% Capítulo 2: Regras gerais de estilo
%%

\mychapter{Cronograma}
\label{Cap:crono}

Este capítulo apresenta considerações de ordem geral sobre a
organização que deve ser adotada no seu documento, tais como número de
páginas, margens e subdivisões.

\section{Passado}
\label{Sec:crono}

Não há um número mínimo ou máximo de páginas para propostas de tema,
dissertações ou teses. Entretanto, se o seu documento for muito menor
do que a média pode transmitir uma idéia de falta de conteúdo a
apresentar. Por outro lado, um documento muito grande corre o risco de
só conseguir a atenção total do leitor no seu início, fazendo com que
as partes mais importantes, que geralmente estão no final do
documento, não sejam devidamente consideradas. Apenas para servir como
parâmetro, estão indicados a seguir os limites usuais quanto ao número
de páginas\footnote{Uma folha corresponde a uma página em impressão em
face simples e a duas páginas em impressão em face dupla} dos
documentos do PPgEE da UFRN, adotando as margens e os espaçamentos
definidos neste modelo:
\begin{itemize}
\item Proposta de tema para exame de qualificação de mestrado:
entre 20 e 40 páginas
\item Proposta de tema para exame de qualificação de doutorado:
entre 30 e 50 páginas
\item Dissertação de mestrado:
entre 50 e 100 páginas
\item Tese de doutorado:
entre 80 e 150 páginas
\end{itemize}

\section{Futuro}

