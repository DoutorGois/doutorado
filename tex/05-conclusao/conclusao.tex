%%
%% Capítulo 5: Conclusões
%%

\mychapter{Conclusões}
\label{Cap:conclusao}

O capítulo final depende do tipo de documento. Nas propostas de tema
deve ser apresentado de forma clara e sucinta o assunto a ser
desenvolvido e o cronograma de execução do trabalho. Nas teses e
dissertações devem ser ressaltadas as principais contribuições do
trabalho e as suas limitações.

As contribuições devem evitar as adjetivações e julgamentos de valor.
Quanto às limitações, não tenha medo de as apresentar: é muito mais
reconhecido um autor que apresenta os casos em que sua proposta não se
aplica do que outro que parece não ter consciência deles.

\section{Encadernação}

As propostas de tema e as versões iniciais das teses e dissertações
são impressas em lado único da folha e em espaçamento um e meio. Para
a encadernação, usa-se geralmente um método simples, tal como espiral
na lateral das folhas e capa plástica transparente. O número de cópias
é igual ao número de membros da banca e pelo menos mais uma (para o
aluno).

As versões finais das teses e dissertações são impressas em frente e
verso e em espaçamento simples. O número mínimo de cópias é o seguinte:
\begin{itemize}
\item 3 cópias para o PPgEE e a UFRN.
\item 1 cópia para cada examinador externo que participou da banca.
\item ao menos 1 cópia para o aluno (não obrigatória).
\item 1 cópia para o orientador (por cortesia, não obrigatória)
\end{itemize}

Para a encadernação, deve-se adotar uma capa rígida de cor azul para
as dissertações de mestrado e de cor preta para as teses de doutorado,
ambas com letras douradas. Na capa deve constar o título do
trabalho, o autor e o ano da defesa. Se possível, a mesma informação
deve ser repetida na lombada do livro.

Para as versões finais, também se exige uma cópia eletrônica (formato
PDF) do texto, bem como outros dados. Maiores informações podem ser
obtidas na página do PPgEE: \url{http://www.ppgee.ufrn.br/}

\section{Para saber mais}

Procure no Google, ora! Brincadeiras a parte, existem inúmeros
tutoriais sobre \LaTeX\ na rede que podem dar maiores informações
sobre o aplicativo. Para conhecer os pacotes disponíveis, uma opção é
o livro \emph{The \LaTeX\ Companion} \cite{LATEX04}, popularmente
conhecido como o ``livro do cachorro''. Outras informações sobre
redação técnica e normas para confecção de teses e dissertações podem
ser encontradas em livros de Metodologia Científica.
