%
% ********** Resumo
%

% Usa-se \chapter*, e não \chapter, porque este "capítulo" não deve
% ser numerado.
% Na maioria das vezes, ao invés dos comandos LaTeX \chapter e \chapter*,
% deve-se usar as nossas versões definidas no arquivo comandos.tex,
% \mychapter e \mychapterast. Isto porque os comandos LaTeX têm um erro
% que faz com que eles sempre coloquem o número da página no rodapé na
% primeira página do capítulo, mesmo que o estilo que estejamos usando
% para numeração seja outro.
\mychapterast{Resumo}


Para explicar a capacidade de navegar no espaço, Edward Tollman postulou a existência de um mapa cognitivo no cérebro.
Desde então busca-se descrever os elementos cerebrais que participam na codificação dos supostos elementos necessários para promover tal capacidade.
Ainda, demonstro a estabilidade do código de velocidade ao longo do tempo e espaço e de contextos.
Demonstro que essa subpopulação possui característica inibitória: eles são inibitórios, apresentam alta frequencia de potenciais de ação e possuem formato de onda curto.
Demonstro que os neurônios pijj 
Caracterizo a natureza fenotípica dessa subpopulação neural através    




A doutrina neural põe o neurônio como a unidade básica do sistema nervoso central, que por sua vez é o sistema responsável pela coordenação do comportamento animal. Os neurônios possuem em sua membrana um complexo proteico que da a capacidade de excitação. Sabe-se que somos capazes de inferir o comportamento animal através de observer a taxa em que o neurônio alterna entre excitação e repouso. Nesse trabalho neurônios temos interesse analisar a entra

Navegação espacial é uma capacidade presente desde os artropodes até os humanos.
Teóricamente essa capacidade depende do cérebro ser capaz de construir uma representação de si e do espaço afim de construir uma memória de sequência.


O sistema mesolimbico foi demonstrado ser necessário para a expressão dessa capacidade.




O resumo deve apresentar ao leitor uma idéia compacta, mas clara do
trabalho descrito na tese. A definição precisa e importância do
problema abordado, os principais objetivos, motivações e desafios da
pesquisa são bons pontos de partida para o resumo. A estratégia ou
metodologia empregada na pesquisa, suas principais contribuições e os
resultados mais importantes também devem fazer parte do resumo. Note
que o resumo não deve ultrapassar uma página.

\vspace{1.5ex}

{\bf Palavras-chave}: Processamento de texto, \LaTeX,
Preparação de Teses, Relatórios Técnicos.
%
% ********** Abstract
%

\mychapterast{Abstract}

The abstract must present to the reader a short, but clear idea of the
work being reported in the thesis. The precise definition and
importance of the problem being addressed, the main objectives,
motivations and challenges of the research are a good starting point
for the abstract. The strategy or metodology employed in the research,
its main contributions, and the most important results achieved may be
part of the abstract as well. Notice that the
abstract must not exceed one page.

\vspace{1.5ex}

{\bf Keywords}: Document Processing, \LaTeX, Thesis Preparation,
Technical Reports.
